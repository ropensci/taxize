\section{Matching species tables with different taxonomic resolution} 

Trait-based approaches are a promising tool in ecology. Unlike taxonomy-based methods, traits may not be constrained to biogeographic boundaries \citep{baird_toward_2011} and have potential to disentangle the effects of multiple stressors \citep{statzner_can_2010}. 

To analyse trait-composition abundance data must be matched with trait databases like \citet{usseglio-polatera_biological_2000}. However these two datatables may contain species information on different taxonomic levels and perhabs data must be aggregated to a joint taxomic level.

taxize can help in this data-cleaning step, providing a reproducible workflow. Here we illustrate this on a small fictitious example.

Suppose we have fuzzy coded trait table with 2 traits with 3 respectively 2 modalities:
\begin{knitrout}
\definecolor{shadecolor}{rgb}{0.969, 0.969, 0.969}\color{fgcolor}\begin{kframe}
\begin{alltt}
(traits <- \hlfunctioncall{read.table}(header = TRUE, sep = \hlstring{";"}, stringsAsFactors = FALSE, text = \hlstring{"taxon;T1M1;T1M2;T1M3;T2M1;T2M2\textbackslash{}nGammarus sp.;0;0;3;1;3\textbackslash{}nPotamopyrgus antipodarum;1;0;3;1;3\textbackslash{}nCoenagrion sp.;3;0;1;3;1\textbackslash{}nEnallagma cyathigerum;0;3;1;0;3\textbackslash{}nErythromma sp.;0;0;3;3;1\textbackslash{}nBaetis sp.;0;0;0;0;0\textbackslash{}n"}))
\end{alltt}
\begin{verbatim}
##                      taxon T1M1 T1M2 T1M3 T2M1 T2M2
## 1             Gammarus sp.    0    0    3    1    3
## 2 Potamopyrgus antipodarum    1    0    3    1    3
## 3           Coenagrion sp.    3    0    1    3    1
## 4    Enallagma cyathigerum    0    3    1    0    3
## 5           Erythromma sp.    0    0    3    3    1
## 6               Baetis sp.    0    0    0    0    0
\end{verbatim}
\end{kframe}
\end{knitrout}


And want to match this to a table with abundances:
\begin{knitrout}
\definecolor{shadecolor}{rgb}{0.969, 0.969, 0.969}\color{fgcolor}\begin{kframe}
\begin{alltt}
(abundances <- \hlfunctioncall{read.table}(header = TRUE, sep = \hlstring{";"}, stringsAsFactors = FALSE, 
    text = \hlstring{"taxon;abundance;sample\textbackslash{}nGammarus roeseli;5;1\textbackslash{}nGammarus roeseli;6;2\textbackslash{}nGammarus tigrinus;7;1\textbackslash{}nGammarus tigrinus;8;2\textbackslash{}nCoenagrionidae;10;1\textbackslash{}nCoenagrionidae;6;2\textbackslash{}nPotamopyrgus antipodarum;10;1\textbackslash{}nxxxxx;10;2\textbackslash{}n"}))
\end{alltt}
\begin{verbatim}
##                      taxon abundance sample
## 1         Gammarus roeseli         5      1
## 2         Gammarus roeseli         6      2
## 3        Gammarus tigrinus         7      1
## 4        Gammarus tigrinus         8      2
## 5           Coenagrionidae        10      1
## 6           Coenagrionidae         6      2
## 7 Potamopyrgus antipodarum        10      1
## 8                    xxxxx        10      2
\end{verbatim}
\end{kframe}
\end{knitrout}



First we do some basic data-cleaning and create a looktable, that will link taxa in trait table with taxa in the abundance table.
\begin{knitrout}
\definecolor{shadecolor}{rgb}{0.969, 0.969, 0.969}\color{fgcolor}\begin{kframe}
\begin{alltt}
\hlcomment{# first we remove ' sp.' from out trait table:}
traits$taxon_cleaned <- \hlfunctioncall{tolower}(\hlfunctioncall{gsub}(\hlstring{" sp."}, \hlstring{""}, traits$taxon))

\hlcomment{# since abundance tables can be very long with repeating taxa, we look}
\hlcomment{# only at unique taxon names This will be a lookup-table linking taxon}
\hlcomment{# names between both tables}
lookup <- \hlfunctioncall{data.frame}(taxon = \hlfunctioncall{tolower}(\hlfunctioncall{unique}(abundances$taxon)), stringsAsFactors = FALSE)
\end{alltt}
\end{kframe}
\end{knitrout}


The we query the taxonomic hierarchy for both tables, since this is the backbone of this procedure:
\begin{knitrout}
\definecolor{shadecolor}{rgb}{0.969, 0.969, 0.969}\color{fgcolor}\begin{kframe}
\begin{alltt}
\hlfunctioncall{require}(taxize)
traits_classi <- \hlfunctioncall{classification}(\hlfunctioncall{get_uid}(traits$taxon_cleaned))
\end{alltt}


{\ttfamily\noindent\itshape\color{messagecolor}{\#\# \\\#\# Retrieving data for species 'gammarus'}}

{\ttfamily\noindent\itshape\color{messagecolor}{\#\# \\\#\# Retrieving data for species 'potamopyrgus antipodarum'}}

{\ttfamily\noindent\itshape\color{messagecolor}{\#\# \\\#\# Retrieving data for species 'coenagrion'}}

{\ttfamily\noindent\itshape\color{messagecolor}{\#\# \\\#\# Retrieving data for species 'enallagma cyathigerum'}}

{\ttfamily\noindent\itshape\color{messagecolor}{\#\# \\\#\# Retrieving data for species 'erythromma'}}

{\ttfamily\noindent\itshape\color{messagecolor}{\#\# \\\#\# Retrieving data for species 'baetis'}}\begin{alltt}
lookup_classi <- \hlfunctioncall{classification}(\hlfunctioncall{get_uid}(lookup$taxon))
\end{alltt}


{\ttfamily\noindent\itshape\color{messagecolor}{\#\# \\\#\# Retrieving data for species 'gammarus roeseli'}}

{\ttfamily\noindent\itshape\color{messagecolor}{\#\# \\\#\# Retrieving data for species 'gammarus tigrinus'}}

{\ttfamily\noindent\itshape\color{messagecolor}{\#\# \\\#\# Retrieving data for species 'coenagrionidae'}}

{\ttfamily\noindent\itshape\color{messagecolor}{\#\# \\\#\# Retrieving data for species 'potamopyrgus antipodarum'}}

{\ttfamily\noindent\itshape\color{messagecolor}{\#\# \\\#\# Retrieving data for species 'xxxxx'}}\end{kframe}
\end{knitrout}


First we look if we get any direct matches between taxon names:
\begin{knitrout}
\definecolor{shadecolor}{rgb}{0.969, 0.969, 0.969}\color{fgcolor}\begin{kframe}
\begin{alltt}
\hlcomment{# first search for direct matches}
direct <- \hlfunctioncall{match}(lookup$taxon, traits$taxon_cleaned)
\hlcomment{# and add the matched name to our lookup table}
lookup$traits <- \hlfunctioncall{tolower}(traits$taxon[direct])
lookup$type <- \hlfunctioncall{ifelse}(!\hlfunctioncall{is.na}(direct), \hlstring{"direct"}, NA)
lookup
\end{alltt}
\begin{verbatim}
##                      taxon                   traits   type
## 1         gammarus roeseli                     <NA>   <NA>
## 2        gammarus tigrinus                     <NA>   <NA>
## 3           coenagrionidae                     <NA>   <NA>
## 4 potamopyrgus antipodarum potamopyrgus antipodarum direct
## 5                    xxxxx                     <NA>   <NA>
\end{verbatim}
\end{kframe}
\end{knitrout}


We found a direct match - \emph{potamopyrgus antipodarum} - so nothing to do here.

Next we look for species which are on a better taxonomic resolution than our in our trait table. 
For these species we'll use the trait-data since no better is available.

\begin{knitrout}
\definecolor{shadecolor}{rgb}{0.969, 0.969, 0.969}\color{fgcolor}\begin{kframe}
\begin{alltt}
\hlcomment{# look for cases where taxonomic resolution in abundance data is higher}
\hlcomment{# than in trait data: here we take the trait-values for the lower}
\hlcomment{# resolution}
\hlfunctioncall{for} (i in \hlfunctioncall{which}(\hlfunctioncall{is.na}(lookup$traits))) \{
    \hlfunctioncall{if} (\hlfunctioncall{is.data.frame}(lookup_classi[[i]])) \{
        matches <- \hlfunctioncall{tolower}(lookup_classi[[i]]$ScientificName) %in% traits$taxon_cleaned
        \hlfunctioncall{if} (\hlfunctioncall{any}(matches)) \{
            lookup$traits[i] <- \hlfunctioncall{tolower}(lookup_classi[[i]]$ScientificName[matches])
            lookup$type[i] <- lookup_classi[[i]]$Rank[matches]
        \}
    \}
\}
lookup
\end{alltt}
\begin{verbatim}
##                      taxon                   traits   type
## 1         gammarus roeseli                 gammarus  genus
## 2        gammarus tigrinus                 gammarus  genus
## 3           coenagrionidae                     <NA>   <NA>
## 4 potamopyrgus antipodarum potamopyrgus antipodarum direct
## 5                    xxxxx                     <NA>   <NA>
\end{verbatim}
\end{kframe}
\end{knitrout}


So our abundance data has two Gammarus species, however trait data is only on Genus level.

Our next step is to search for species were we have to aggregate tait-data, since our abundance data is on a lower taxonomic level.
We are walking the taxomomic latter for the species in our trait-data upwards and search for matches with out abundance data. Since we'll have many taxa in the trait-data belonging to one taxon, we'll take the median modality scores as an approximation. Of course also other methods may be used here, e.g. weighting by genetic distance.


\begin{knitrout}
\definecolor{shadecolor}{rgb}{0.969, 0.969, 0.969}\color{fgcolor}\begin{kframe}
\begin{alltt}
\hlcomment{# look for cases taxonomic resolution in abundance data is lower than in}
\hlcomment{# trait data, here we need to aggregate the trait-values (eg. median value}
\hlcomment{# for modality)}

\hlfunctioncall{for} (i in \hlfunctioncall{which}(\hlfunctioncall{is.na}(lookup$traits))) \{
\hlcomment{    # find matches}
    agg <- \hlfunctioncall{sapply}(traits_classi, \hlfunctioncall{function}(x) \hlfunctioncall{any}(\hlfunctioncall{tolower}(x$ScientificName) %in% 
        lookup$taxon[i]))
    \hlfunctioncall{if} (\hlfunctioncall{sum}(agg) > 1) \{
\hlcomment{        # add taxon as aggregate to trait-table}
        traits <- \hlfunctioncall{rbind}(traits, \hlfunctioncall{c}(\hlfunctioncall{paste0}(lookup$taxon[i], \hlstring{"-aggregated"}), \hlfunctioncall{apply}(traits[agg, 
            2:6], 2, median), \hlfunctioncall{paste0}(lookup$taxon[i], \hlstring{"-aggregated"})))
\hlcomment{        # fill lookup table}
        lookup$traits[i] <- \hlfunctioncall{paste0}(lookup$taxon[i], \hlstring{"-aggregated"})
        lookup$type[i] <- \hlstring{"aggregated"}
    \}
\}
lookup
\end{alltt}
\begin{verbatim}
##                      taxon                    traits       type
## 1         gammarus roeseli                  gammarus      genus
## 2        gammarus tigrinus                  gammarus      genus
## 3           coenagrionidae coenagrionidae-aggregated aggregated
## 4 potamopyrgus antipodarum  potamopyrgus antipodarum     direct
## 5                    xxxxx                      <NA>       <NA>
\end{verbatim}
\end{kframe}
\end{knitrout}


Finally we have one taxon left. We remove this from our dataset:
\begin{knitrout}
\definecolor{shadecolor}{rgb}{0.969, 0.969, 0.969}\color{fgcolor}\begin{kframe}
\begin{alltt}
abundances <- abundances[!abundances$taxon == lookup$taxon[\hlfunctioncall{is.na}(lookup$traits)], 
    ]
\end{alltt}
\end{kframe}
\end{knitrout}



No we can create species x sites and traits x species matrices, which could be plugged into methods to analyse trait responses \citep{kleyer_assessing_2012}.


\begin{knitrout}
\definecolor{shadecolor}{rgb}{0.969, 0.969, 0.969}\color{fgcolor}\begin{kframe}
\begin{alltt}
\hlcomment{# species (as matched with trait table) by site matrix}
\hlfunctioncall{require}(reshape2)
\end{alltt}


{\ttfamily\noindent\itshape\color{messagecolor}{\#\# Loading required package: reshape2}}\begin{alltt}
\hlcomment{# reshape data to long format and name rows by samples}
L <- \hlfunctioncall{dcast}(abundances, sample ~ traits, fun.aggregate = sum, value.var = \hlstring{"abundance"})
\end{alltt}


{\ttfamily\noindent\bfseries\color{errorcolor}{\#\# Error: undefined columns selected}}\begin{alltt}
\hlfunctioncall{rownames}(L) <- L$sample
\end{alltt}


{\ttfamily\noindent\bfseries\color{errorcolor}{\#\# Error: object 'L' not found}}\begin{alltt}
L$sample <- NULL
\end{alltt}


{\ttfamily\noindent\bfseries\color{errorcolor}{\#\# Error: object 'L' not found}}\begin{alltt}
L
\end{alltt}


{\ttfamily\noindent\bfseries\color{errorcolor}{\#\# Error: object 'L' not found}}\begin{alltt}

\hlcomment{# traits by species matrix}
Q <- traits[, 2:7][\hlfunctioncall{match}(\hlfunctioncall{names}(L), traits$taxon_cleaned), ]
\end{alltt}


{\ttfamily\noindent\bfseries\color{errorcolor}{\#\# Error: object 'L' not found}}\begin{alltt}
\hlfunctioncall{rownames}(Q) <- Q$taxon_cleaned
\end{alltt}


{\ttfamily\noindent\bfseries\color{errorcolor}{\#\# Error: object 'Q' not found}}\begin{alltt}
Q$taxon_cleaned <- NULL
\end{alltt}


{\ttfamily\noindent\bfseries\color{errorcolor}{\#\# Error: object 'Q' not found}}\begin{alltt}
Q
\end{alltt}


{\ttfamily\noindent\bfseries\color{errorcolor}{\#\# Error: object 'Q' not found}}\begin{alltt}

\hlcomment{# check}
\hlfunctioncall{all}(\hlfunctioncall{rownames}(Q) == \hlfunctioncall{colnames}(L))
\end{alltt}


{\ttfamily\noindent\bfseries\color{errorcolor}{\#\# Error: object 'Q' not found}}\end{kframe}
\end{knitrout}


